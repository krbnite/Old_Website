% mainfile: ../hw2.tex
\textbf{Summarize a science ``highlight'' from the TIMED mission. Cite
references.}

One science highlight resulting from the TIMED mission was the
observation that the Earth's thermosphere breathes in sync with the
passage of corotating interaction regions [CIRs] and, thus, in sync with solar
coronal holes and solar rotation. Looking at a 5-year period between
2002 to 2006, \cite{Mlynczak2008} used time series data from the SABER to examine
daily global power radiated by carbon dioxide ($15 \mu m$)
and nitric oxide (at $5.3 \mu m$) between 100-200 [km] altitude, and time
series data from the SEE instrument to examine the daily absorbed ultraviolet power
within the same altitude region. Spectral analysis of the time series
showed the presence of a statistically significant 9-day periodicity in
the SABER (geo-radiation) data, but no such periodicity in the SEE
(solar UV radiation) data. Further investigation into geomagnetic
indices revealed a corresponding 9-day periodicity, which the authors
linked to three recurrent coronal holes on the Sun. 

Observations from the CHAMP satellite (\cite{Thayer2008};
\cite{Lei2008}) confirm these results.
