% mainfile: ../hw3.tex
\textbf{Estimate typical proton and electron gyro-radii and bounce
period for a 100-keV particle.}

Let $q := |electron charge|$. Then we can write down the electron and
proton gyroradii:
  \begin{align*}
    r_{p} = \frac{m_{p}v_{{p}}}{qB} \\
    r_{e} = \frac{m_{e}v_{{e}}}{qB}
  \end{align*}

More generally in radiation belt physics, attention must be paid to the
relativistic forms of such equations since particle speeds are often
relativistic. The equation then take the form:

\begin{align}
  r_{p} = \frac{p_{{p}}}{qB} \\
  r_{e} = \frac{p_{{e}}}{qB}
\end{align}

We know the kinetic energy, so we can find the momentum in the relativistic
setting by the following equation relating kinetic energy and momentum:

\begin{align}
  p =\frac{1}{c} \sqrt{(E_{k}^{2}+mc^{2}) - m^{2}c^{4}}
\end{align}


We can compute these values for an electron and proton and plug them
into equations (15)-(16):

\begin{gather*}
  m_{e} \approx 0.51 \quad[MeV/c^{2}] \\
  m_{p} \approx 938 \quad[Mev/c^{2}] \\
  p_{e} =\sqrt{ (0.1+0.51)^{2} - (0.51)^{2} } \quad[MeV/c] =
  0.335 \quad[MeV/c]  \\
  p_{p} =\sqrt{ (0.1+938)^{2} - (938)^{2} } \quad[MeV/c] = 13.7
  \quad[MeV/c] \\
  r_{e} = \frac{0.335 [MeV/c]}{qB} \\
  r_{p} = \frac{13.7 [MeV/c]}{qB}
\end{gather*}

The dipole approximation for the geomagnetic field is good enough for
close enough trapped energetic particles. The equation is given by:

\begin{align}
  B = \frac{B_{0}}{R^{3}}\sqrt{1+3sin^{2}\lambda}\\
\end{align}

$B_{0}$ is reference value of Earth's field at the surface; a typical
number is 30,000 [nT]. R is given in Earth radii, i.e., $R=r/R_{E}$. The
angle, $\lambda$, represents latitude.

To crunch a number we will look at particles passing through the
equatorial plane at $L=2$, which corresponds to $r=2$ and $\lambda=0$.
This gives us a magnetic field strength of 3750 [nT].

Units: 
\begin{gather*}
1 T = \frac{kg}{C*s}\\
1 eV = q \frac{kg*m^{2}}{C*s^{2}}
\end{gather*}
So...

\begin{gather*}
  r_{e} = \frac{p_{p}}{qB} = \frac{0.335*10^{6}*q
  [\frac{kg*m^{2}}{C*s^{2}}]}{(3750*10^{-9})(3*10^{8})q
  [\frac{kg*m}{C*s^{2}}]}\\
  \quad\quad = \frac{0.335}{3*3750}10^{7}[m] = 298  [m]\\
  \\
  r_{p} = \frac{13.7*10^{6}q
  [\frac{kg*m^{2}}{C*s^{2}}]}{(3750*10^{-9})(3*10^{8})q
  [\frac{kg*m}{C*s^{2}}]}\\
  \quad\quad = \frac{13.7}{3*3750}10^{7} [m] = 1.22*10^{4} [m]
\end{gather*}

Given the nature of these being estimates, let's just say that the
electron gyradius out around $L=2$ in the equatorial plane is about 300
[m] and the proton gyroradius is a cool 10 [km]. 

\textbf{The Bounce Periods}

Assuming an equatorial pitch angle and L shell, and having values for
earth's radius, the particle energy, and the particle mass, we can
approximate the bounce period with the following equation (source:
http://how.gi.alaska.edu/ao/msp/chapters/chapter5.pdf):

\begin{gather*}
  \tau_{b}\approx \frac{LR_{E}}{\sqrt{E/m}}(3.7-1.6sin\alpha_{eq})
\end{gather*}

We will say that $L=2$ and $\alpha_{eq}=45^{\circ}$. Then:

\begin{align*}
  \tau_{b,e} = \frac{2*6371
  [km]}{\sqrt{0.1c^{2}/0.51}}(3.7-\frac{1.6}{\sqrt{2}})\\
  \approx 0.25 [s]
   \\
   \tau_{b,p} = \frac{2*6371
   [km]}{\sqrt{0.1c^{2}/938}}(3.7-\frac{1.6}{\sqrt{2}})\\
  \approx 1000 [s] \\
  \approx 16.7 [min]
\end{align*}
