So far these notes are just from Wikipedia; open Griffiths and Jackson and get
some more info
--------------------------------------------------

* the Poynting vector represents the directonal energy flux density (the rate
of energy transfer per unit area, in units of Watts per square-meter) of an
electromagnetic field

* the usual ("Abraham") form of the Poynting vector is:
$\vec{S}=\vec{E}\times\vec{H}$
* but under the right conditions, this simplifies to:
$\vec{S}=\frac{1}{\mu_{0}}\vec{E}\times\vec{B}
* actually, there is also a Minkowski form of the Poynting vector as well,
* which in free space looks like:
$S=\frac{1}{\eps_{0}\mu_{0}}\vec{D}\times\vec{B}=c^{2}\vec{D}\times\vec{B}$
-- in natural units, where c=1, it is not clear which form of the Poynting
vector (Abraham or Minkowski) should be considered the true vector
representing electromagnetic energy flow; Pfeifer [2001], "Momentum of an
electromagnetic wave in dielectric media," summarizes the controversy, and
also helps resolve it a bit (read paper and report details)


Poynting's Theorem
* note that the theorem below pertains only to linear, nondispersive, and
uniform materials (i.e., if the constitutive relations can be written with
constant permissivity and permeability)
-- a generalization to dispersive media is possible (at the cost of extra
terms and a loss of a clear physical interpretation)
Let $u=\frac{1}{2}\left(\vec{E}\vdot\vec{D}+\vec{B}\vdot\vec{H}\right)$; this
represents the electromagnetic energy density (a function of space and time).
We have a conservation law like so:
$\frac{\partial{u}}{\partial{t}} = -\nabla\vdot\vec{S} -
\vec{J}_{f}\vdot\vec{E}$
-- $\vec{J}_{f}$ is the current density of free charges
-- the first term on the RHS represents the net electromagnetic energy flow
into a small volume, while the second term represents the subtracted portion
of the work done by free electrical currents that are not necessarily
converted into electromagnetic energy (dissipation, heat). In this definition,
bound electrical currents are not included in this term, and instead
contribute to $\vec{S}$ and u.

------

For ionospheric calculations of Poynting flux, there are various types of 
Poynting flux calculations that are done: for example, one can look at
Ex$\Delta$B, where $\Delta$B is the magnetic field perturbation due to
field-aligned currents [FACs]. So in this case, we are looking at the
FAC-associated Poynting flux and ignored the Poynting flux associated with the
quasi-static main geomagnetic field. Alternatively, one can look at Poynting
flux as a function of frequency, or the Poynting flux of a given frequency
band, e.g., one can compute the Poynting flux due to the ULF Pc5 frequency
band.  






REFERENCES
Huang, Cheryl. Personal Communication (2015-Jan-27).
* look at Griffiths and Jackson
