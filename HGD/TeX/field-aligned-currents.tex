From Cheryl Huang [2004]:
* Vasyliunas [1970] showed that FACs flow into or out of the ionosphere
whenever gradients of the magnetospheric plasma pressure, p, are not aligned
with gradients in magnetic flux tube volume, $\xi=\int{\frac{ds}{B}}$: 
$ j_{\par} = -\frac{B_{m}}{2B_{e}}\vec{b_{e}}\vdot\left[
\nabla\times\nabla \int{\frac{ds}{B}} \right] $
-- where \vec{b_{e}} is a unit vector along Earth's magnetic field at the 
   equator ($B_{e}$) and $B_{m}$ is the magnetic field at the top of the
ionosphere
-- sooo, in the inner magnetosphere near the dusk (dawn) meridians where
$\nabla\xi$ is mostly radial, we expect FACs into (out of) the ionosphere in
the evening (morning) sector due to the eastward (westward) pressure
gradients

* up to the 1970s, the existence of FACs were hotly debated (for a historical
summary, see: Dessler [1984]: The evolution of arguments regarding the
existence of field-aligned currents; for the papers that established the
existence of FACs: Ijima and Potemra [1976, 1978]; predicted to exist by
Birkeland, I beleve -- see: Egeland [1984]: Kristin Birkeland, the man and the
scientist)

* Ijima and Potemra [1976,1978] established that global distributions of FACs
indeed exist; the currents were named Region 1 [R1] and Region 2 [R2] FACs. 
*R1 currents are found near the poleward boundary of the auroral oval; the R1
currents flow into the ionosphere on the dawnside and out on the duskside
* R2 currents are equatorward of the R1 currents and have opposite polarities:
they flow into the ionosphere on the duskside and out on the dawnside


Issues and debates surrounding FACs
-- much of this is ripped directly from Cheryl's paper
* What are the physical requirements and implications for FACs as the agents
responsible for the coupling of the ionosphere to the  magnetosphere?
Vasyliunas [1970] used force balance and current continuity to show that FACs
are the consequence of mislalignments between plasma pressure gradients and
flux tube volume gradients in the magnetosphere (eq'ns shown above)
 -- tidbit: the RCM model implements Vasyliunas' prescription
* Ohm's law dictates that FACs control distributions of electric potential in
the ionosphere; indirectly, through magnetic mapping, they also control
electric field distributions in the magnetosphere
* Ground signatures of FACs: Fukushima [1976] developed a theorem
demonstrating that if FACs close through poloidal currents flowing in a
uniformly-conducting ionosphere, magnetic perturbations at ground level
exactly cancel and, thus, no ground signatures of the FACs are produced
(Fukushima's theorem)


empirical formual for specifying fac distributions
richmond and kamide [1988] developed an empirical approach to specify fac and
ionospheric potential distributions using ground- and space-based data; these
 empirical formulas were refined and developed, ultimately culminating in what
is known as assimilative mapping of ionospheric electrodynamics [amie] (cite:
richmond [1992]). amie inverts magnetic perturbations observe by mid-to-high
latitude ground stations to specify equivalent ionospheric currents. however,
these inversions are limited... raeder et al [2001] argued that translating
equivalent currents into realistic electric field estimates requires precise
knowledge about the distribution of ionospheric conductances. this concern can
be remedied a bit, though, since ionospheric conductances can be derived given
knowledge of solar ultraviolet radiances [wallis and budzinski, 1981] and the
fluxes of energetic particles precipitating from the magnetosphere [robinson
et al, 1987]. however, given an accurate description of ionospheric
conductances, there is still the issue of fukushima's theorem: that
perturbations produced by facs and poloidal closing currents exactly cancel on
the ground. this means that, inherently, amie is vulnerable to underestimating
fac intensities and ionospheric electric fields!



Analyzing FACs using DMSP spacecraft
Since DMSP are polar-orbiting, sun-synchronous satellites, if we assume the
field-aligned current sheets are spread azimuthally, then at the DMSP altitude
of ~840 km, if there are such field-aligned current sheets in the vicinity of
the 2100/0900 or 1800/0600 geographic local time meridians, the particular
DMSP spacecraft can be thought to pass through the sheet at normal incidence.
In spacecraft-centered coordinates (X: nadir-directed, Y: velocity-directed,
Z: XxY), this means that the FAC is directed approximately along X (i.e., if
passing through a R2 FAC at 0600, the current is in the -X direction headed
towards the magnetosphere; if passing the R1 FAC at 0600, the FAC is in the +X
direction headed down into the ionosphere).
* two governing equations for analyzing combined SSM and SSIES data are
Ampere's law and current continuity:
(i) $j_{\par}=\frac{1}{\mu_{0}}
\left[\nabla\times\delta\vec{B}\right]_{\par}$
(ii) $\frac{\partial{j_{\par}}}{\partial{s}} =
-\nabla_{\perp}\vdot\vec{j}_{\perp}$
-- s represents an infinitesimal distance along the field line
* in the ionosphere, the current density perpendicular to the Earth's
magnetic field $\vec{B}_{E}$ is
$\vec{j}_{perp}=\sigma_{P}\vec{E}-\sigma_{H}(\vec{E}\times\vec{b})$
-- where the sigmas are the Pedersen and Hall conductivities, $\vec{b}$
is a unit vector along $\vec{B}_{E}$
* we can integrate the current continuity equation, (ii), from the
satellite location to the bottom of the ionosphere (along the field
line, I think) to get:
$j_{\par}=\nabla_{\perp}\vdot\vec{I}_{\perp} =
\nabla_{\perp}\vdot\left[\Sigma_{P}\vec{E}-\Sigma_{H}(\vec{E}\times\vec{b})\right]$
-- the capital sigmas are the Pedersen and Hall conductances

In the spacecraft-centered coordinates for a DMSP satellite, $j_{\par}$ can be
written $j_{X}$ and $j_{\perp}$ can be written $j_{Y}$.




