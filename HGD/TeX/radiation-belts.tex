% vim:tw=72 sw=2 ft=tex spell spelllang=en
% mainfile: ../hw1.tex
\textbf{What are the Earth’s radiation belts, and how are the formed?}
 
The Earth's radiation belts are torus-like structures surrounding the
Earth in the inner magnetosphere made up of trapped energetic particles.
Naturally, there are usually only two of these structures, referred to
as the "inner" and "outer" radiation belts, although there are
additional (natural) radiation belts at times --- made famous by
observations from the Van Allen Probes mission. dditional radiation
belts
have also been formed by nuclear detonations in Earth's upper atmosphere. 

The outer belt is mostly comprised of dynamic, extremely energized
electrons, whereas the inner belt is mostly comprised of a mixture of
energized electrons and protons. Ion populations also coexist in these
regions. Although we refer to the inner region as the
"inner belt", it has a finer structure made up of a proton belt and an
electron belt. Moreover, other ions form belt populations as
well; according to world-famous Dr. Andrew Gerrard, an
interesting, oft-ignored, "red headed stepchild" helium belt structure
exists and is important in the vertical coupling of the
ionosphere/thermosphere system to the inner magnetosphere
\citep{Gerrard2014}.

How do the the belts form? Where do the energetic particle populations
come from? Cosmic rays, solar wind events, interaction with Earth's upper
atmosphere --- while all partially responsible for the formation and energization
of the belts, their formation, their depletion, tranport, etc, these are all still
active areas of research.
