% mainfile: ../hw4.tex
\textbf{In lecture 13, slide 18, an estimate of the ionospheric electric
field is made. Verify the calculation and list any assumptions.}

The assumptions begin with an electrostatic electric field -- that is
the curl of the electric field vanishes everywhere, which is just a
fancy way of saying that the magnetic field is unchanging in time.

This assumption is encapsulated by the following form of Faraday's law:

\begin{gather*}
  \frac{\partial \vec{B}}{\partial t} = -\bigtriangledown\times\vec{E} = 0
\end{gather*}

 Coupled with our demand that the electric and magnetic fields must be
 perpendicular, this amounts to assuming no electromagnetic
disturbances propagating anywhere! For example, assume plane waves:

\begin{gather*}
  \frac{\partial \vec{B}}{\partial t} = -i\omega\vec{B}
  \rightarrow \omega=0\\
  \bigtriangledown\times\vec{E} = i\vec{k}\times\vec{E}
  \rightarrow \vec{k} || \vec{E}\\
  \bigtriangledown\times\vec{B} = i\vec{k}\times\vec{B} =
  \frac{1}{v_{phase}^{2}} \frac{\partial \vec{E}}{\partial t} =
  \frac{-i\omega}{v_{phase}^{2}}\vec{E} \\
  \rightarrow \frac{k}{E}\vec{E}\times\vec{B} = const*\vec{E}\\
  \rightarrow k=0
\end{gather*}

Since we assume the the electric and magnetic fields are everywhere
perpendicular, then k=0 must be true. No wave propagation.

From the ever-present vanishing divergence of the magnetic field and, in
this case, a time-stationary magnetic field, we know that the magnetic
flux through an area, A, is constant in time. Say this area, A, is in
the ionosphere. If we trace the field lines at the boundary of A,
$\partial A$, out into the magnetosphere, the amount of flux will be
constant, although the area will have changed. Let's say A is a circle
whose radius is a characteristic length of the ionosphere, and that its
mapping into the magnetosphere is approximately circular with a radius
characteristic of the magnetosphere.

\begin{gather*}
  \int_{iono}{\vec{B}\cdot d\vec{A}} \quad\approx\quad 
  2\pi B_{iono}L_{iono}^{2}\\
  \int_{msph}{\vec{B}\cdot d\vec{A}} \quad\approx\quad 
  2\pi B_{msph}L_{msph}^{2}
\end{gather*}

Here, $B_{msph}$ and $B_{iono}$ are the average values of the magnetic
field in their respective slices of the flux tube. Since we know that
the total flux is the same in both regions of the flux tube, we then
know that the average field in the magnetosphere slice of the tube must
be smaller than the average field in the ionosphere slice to account for
the larger characteristic length. 

At any rate, we have the equation:

\begin{gather*}
  B_{msph}L_{msph}^{2} = B_{iono}L_{iono}^{2}
\end{gather*}

To calculate the electic field, we can compute the change of voltage
over the change in distance.

\begin{gather*}
  E_{iono} = \frac{\bigtriangleup \phi}{L_{iono}}\\
  E_{msph} = \frac{\bigtriangleup \phi}{L_{msph}}
\end{gather*}

We don't know the potentials, but if we take the ratio, we can put it in
terms of what we already derived above:

\begin{gather*}
  \frac{E_{iono}}{E_{msph}} = \frac{L_{msph}}{L_{iono}} =
  \sqrt{\frac{B_{iono}}{B_{msph}}}
\end{gather*}

And finally to get the velocity ratios in the lecture slides, use the
magnitude of Lorentz force equation (assume motion of particle
perpendicular to magnetic field) and typical MHD assumptions:

\begin{gather*}
  \vec{F}/q = \vec{E} + \vec{v}\times\vec{B} = 0
  \rightarrow E = vB
  \rightarrow v = E/B
\end{gather*}

Then we can take the ratios and plug in things we know:

\begin{gather*}
  \frac{v_{iono}}{v_{msph}} =
  \frac{E_{iono}/B_{iono}}{E_{msph}/B_{msph}} =
  \frac{E_{iono}}{E_{msph}}\frac{B_{msph}}{B_{iono}} =
  \frac{B_{msph}}{B_{iono}}\sqrt{\frac{B_{iono}}{B_{msph}}} =
  \sqrt{\frac{B_{msph}}{B_{iono}}}
\end{gather*}

To estimate the electric field, assume we know the typical electric
field value in the magnetosphere, the ratio of the characteristic
lengths of the magnetosphere and ionosphere, and the typical speed of a
particle in the magnetosphere:

\begin{gather*}
  E_{msph} = 0.7 mV/m
  L_{iono}/L_{msph} = 1/60 \\
  v_{msp} = 60 km/x
\end{gather*}

Then we can rearrange the above equations to look like:

\begin{gather*}
  E_{iono} = \frac{L_{msph}}{L_{iono}}E_{msph}
  = 60*0.7 [mV/m] = 42 [mV/m]
\end{gather*}

Hmm... Actually we didn't even need to assume we knew the particle
velocity in the magnetosphere for that one. 
