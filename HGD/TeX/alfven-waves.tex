This derivation has something to do w/ Alfven waves: 
% mainfile: ../hw2.tex
\textbf{Reproduce the equations from Lecture 4, slides 18 to 19.}

We start with the dispersion relation on slide 17:

\vspace{5mm}

$ -\omega^{2} \vec{V}_{1} +
(v_{s}^{2}+v_{A}^{2})(\vec{k}\cdot\vec{V_{1}})\vec{k}+(\vec{k}\cdot\vec{v_{A}})[(\vec{k}\cdot\vec{v_{A}})\vec{V_{1}}-(\vec{v_{A}}\cdot\vec{V_{1}})\vec{k}-(\vec{k}\cdot\vec{V_{1}})\vec{v_{A}}]=0$

\vspace{5mm}

We also note that $\vec{V_{1}}$ is a plane wave:

\vspace{5mm}

\begin{center}
  $\vec{V}_{1}(\vec{r},t) =
  \vec{V}_{1,0}exp[i(\vec{k}\cdot\vec{r}-\omega t)]$
\end{center}

\vspace{5mm}

We want to look at situation where $\vec{k}\cdot\vec{B_{0}}=0$. This is
the same as saying $\vec{k}\cdot\vec{v_{A}}=0$ since the Alfven velocity
is defined as $\vec{v_{A}}:=\vec{B}/\sqrt{\mu_{0}\rho_{0}}$.

Well this greatly simplifies the stated dispersion relation: the big,
scary square-bracketed term is multiplied by zero and vanishes!

\vspace{5mm}

\begin{center}
$-\omega^{2} \vec{V}_{1} + (v_{s}^{2}+v_{A}^{2})(\vec{k}\cdot\vec{V_{1}})\vec{k}=0$
\end{center}

\vspace{5mm}

Now, to solve for $\vec{V_{1}}$, it just a simple rearrangement of this
last equation:

\vspace{5mm}

\begin{center}
$-\omega^{2} \vec{V}_{1} = -(v_{s}^{2}+v_{A}^{2})(\vec{k}\cdot\vec{V_{1}})\vec{k}$

$\vec{V}_{1} = (v_{s}^{2}+v_{A}^{2})(\vec{k}\cdot\vec{V_{1}})\vec{k}/\omega^{2} $
\end{center}

\vspace{5mm}

Noting that the direction the wave is travelling in is the direction of
the wave vector, we can write: $\vec{k}||\vec{V}_{1}$. This allows us to
write:

\vspace{5mm}

\begin{center}
  $\vec{V}_{1} = (v_{A}^{2}+v_{s}^{2})\frac{kV_{1}}{\omega^{2}}\vec{k}$
\end{center}

\vspace{5mm}

...which, again becuase of the parallel nature of $\vec{V}_{1}$ and
$\vec{k}$, reduces to:

\vspace{5mm}

\begin{center}
  $1 = (v_{A}^{2}+v_{s}^{2})\frac{k^{2}}{\omega^{2}}$
\end{center}
\vspace{5mm}

Or:

\vspace{5mm}
\begin{center}
  $\frac{\omega}{k}=\sqrt{v_{s}^{2}+v_{A}^{2}}$
\end{center}
\vspace{5mm}

For the magnetic field derivations on slide 19, we use equation (***)
from slide 16:
\vspace{5mm}
\begin{center}
  $\frac{\partial \vec{B}}{\partial t} - \bigtriangledown \times (\vec{V}_{1}
  \times \vec{B}_{0}) = 0$
\end{center}
\vspace{5mm}

We assume that $\vec{B}$ is also a plane wave, and so our differential
equation becomes algebraic:

\vspace{5mm}
\begin{center}
  $\frac{\partial}{\partial t} \mapsto -i\omega$ \\
  $\bigtriangledown \mapsto i\vec{k}$
\end{center}
\vspace{5mm}

Using these mappings, the fact that $\vec{V}_{1}$ is parallel to
$\vec{k}$ which is perpendicular to $\vec{B}_{0}$, and the trusty "BAC
CAB" vector triple product rule, our equation
simplifies:

\vspace{5mm}
\begin{center}
  $-\omega \vec{B}_{1} - \vec{V}_{1}(\vec{k}\cdot\vec{B}_{0}) +
  (\vec{k}\cdot\vec{V}_{1})\vec{B}_{0} = 0$ \\
  $-\omega \vec{B}_{1} + kV_{1}\vec{B}_{0} = 0$ \\
  $\vec{B}_{1} = \frac{kV_{1}}{\omega}\vec{B}_{0}$ \\

\end{center}
\vspace{5mm}

And so: 
  $B_{1} = \frac{kV_{1}}{\omega}B_{0}$




This derivation also has something to do w/ Alfven waves: 
% mainfile: ../hw2.tex
\textbf{For $V_{A}<V_{S}$, compare the shear Alfven wave to the slow
MHD wave for an arbitrary $\theta$}.  

The dispersion relation for the shear Alfven wave, when $\theta$ is 0,
reduces to:

\begin{center}
\begin{math}
  (\frac{\omega}{k})^{2} = \frac{1}{2}(v_{s}^{2}+v_{A}^{2}) \pm
  \frac{1}{2}(v_{s}^{2}+v_{A}^{2})
\end{math}
\end{center}

The trivial solution is $\omega = 0$. The nontrivial solution:
$(\frac{\omega}{k})^{2} = (v_{s}^{2}+v_{A}^{2})$ 

If the sound speed in the medium is much smaller than the Alfven speed, then we have the
dispersion relation for the slow MHD wave:

\begin{center}
\begin{math}
  \frac{\omega}{k} = v_{A}
\end{math}
\end{center}

So the shear Alfven wave is a generalization of the slow MHD wave to
arbitrary angle in media where the sound speed is comparable to the
Alfven speed.




% mainfile: ../hw2.tex
\textbf{Sometimes I see the Alfven wave speed written as
$V_{A}=\frac{B}{\sqrt{4\pi\rho}}$ and sometimes as
$V_{a}=\frac{B_{SW}}{\sqrt{\mu_{0}m_{i}n}}$. Why?}

Lots of things here. First the small stuff:

\begin{center}
  \begin{enumerate}
    \item $V_{a}$ and $V_{A}$ are two choices of representation for the same dang
      thing.
    \item Specifying $B_{SW}$ means you were reading something about the solar
      wind. Specifying just B means you could have been reading any number of
      sources about MHD in general.
    \item $m_{i}n$ is another way of writing $\rho_{i}$, which is just
      specifying that the mass density $\rho$ is almost fully described by the
      ion mass density (as opposed to the electron mass density); in practice
      then, $\rho$ and $\rho_{i}$ refer to the same thing.
  \end{enumerate}
\end{center}

The bigger stuff (probably the stuff you were actually referring to):

\begin{center}
  \begin{enumerate}
    \item Gaussian Units: $V_{A}=\frac{B}{\sqrt{4\pi\rho}}$
    \item SI Units: $V_{a}=\frac{B_{SW}}{\sqrt{\mu_{0}m_{i}n}}$
  \end{enumerate}
\end{center}
