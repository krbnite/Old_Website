% mainfile: ../hw4.tex
\textbf{What is the LLBL? Describe what causes it.}

The low-latitude boundary layer (LLBL) is the region where
tailward-flowing magnetosheathlike plasma is found on the magnetospheric side of the
magnetopause current layer at low geomagnetic latitudes. The LLBL can
exist on both open or closed field lines. Its thickness increases with
increasing distance from the subsolar point and, also, during times of
northward interplanetary magnetic field [IMF]. 

Several types of mechanisms describing the formation of the LLBL have been
suggested: those invoking magnetic reconnection, those involving
impulsive penetration of magnetosheath plasma, and viscous interaction
mechanisms.

\citet{scholer1997}
