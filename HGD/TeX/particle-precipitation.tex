
* usually refers to particle precipitation over the polar cap and in the
auroral regions (but I suppose it doesn't need to)
* the varios regions and types of particle precipitation are important
to understand b/c they map out to different regions of the magnetosphere
(e.g., outer radiation belt, ring current, trapping zone, quasi-trapping
zone, etc, etc)
* Winningham and Heikkila [1974] showed that the polar cap is
characterized by nearly uniform precipitation of electrons with E < 100
eV; this type of precipitation is now known as "Polar Rain."
* Winningham [1975] distinguished between two types of auroral electron
precipitation: boundary plasma sheet [BPS] precipitation and central
plasma sheet [CPS] precipitation
 -- BPS precipitation occurs in the poleward part of the nigthsdie oval
 and is highly structured in energy-versus-time spectrograms
 -- CPS precipitation occurs in the equatorward part of the oval and
 varies slowly with latitude
* DeCoster and Frank [1979] employed similar nomenclature to describe
observed particle populations in the "plasma sheet boundary layer" and
the "central plasma sheet"
 -- the central plasma sheet precipitation
coincides with that described above
 -- the PSBL is a region of intense electric fields and
 velocity-dispersed ion fluxes; it does not exactly correspond to the BPS 
 precipitation described above since it covers a much narrower range of 
 magnetic latitudes than the BPS.
 * Newell [1996] described nightside morphology of particle
 precipitation (MORE ON THIS), which extended work by Gussenhoven et al
 [1983] who interpreted the equatorward boundary of auroral electrong
 precipitation as mapping to the zero-energy Alfven boundary in the
 magnetosphere
 * Boundaries specified in Newell [1996] include (NOT YET COMPLETE /
 COMPLETE THIS LIST), from low-to-high latitude:
 B1: equivalent to the instantaneous zero-energy Alfven boundary
 B2e: marks entry into the main plasma sheet, where average electron
 energies cease to increase with magnetic latitude
 B2i: the ion isotropy boundary: marks the transition between
 quasi-dipolar and stretched field lines
 B5: poleward boundary of the main oval



