% mainfile: ../hw5.tex
\textbf{What is diffusivity, as applied to magnetospheric magnetic and
electric diffusivity?}

Louis Lanzerotti wrote the book on particle diffusion in the
magnetosphere -- radial diffusion,
pitch angle diffusion, etc. Associated with each type of particle
diffusion is a diffusion constant, often called the "diffusivity" when
modeling diffusion for a system. With this in mind, in Lanzerotti's
scheme, since both the electric and magnetic fields drive the particle
diffusion, one might refer to a "magnetic diffusivity" and an "electric
diffusivity" when describing particle diffusion, e.g., "radial magnetic
diffusivity." However, one often finds the phrase
"magnetic diffusivity" reserved for a standard parameter in plasma physics 
that quantifies how magnetic field lines diffuse through a plasma; this
is in distinction to the "magnetic diffusivity" in Lanzerotti's scheme, which 
quantifies how particles diffuse through a magnetic field. To avoid this
confusion, many researchers refer to the magnetically- and
electrically-induced particle diffusivities and "diffusion
coefficients." But one should be wary since this is not always the case.
