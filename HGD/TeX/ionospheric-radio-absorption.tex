\documentclass[12pt,a4paper]{article}

% % % % % % % % % % % % % % % % % % % % % % % % % % % % % % % % % % % % % % % % % % % % % % % % 
% % % % % % % % % % % % % % % % % % % % % % % % % % % % % % % % % % % % % % % % % % % % % % % % 
% % % % % % % % % % % % % % % % % % % % % % % % % % % % % % % % % % % % % % % % % % % % % % % % 
% % % % % % % % % % % % % % % % % % % % % % % % % % % % % % % % % % % % % % % % % % % % % % % % 
% % % % % % % % % % % % % % % % % % % % % % % % % % % % % % % % % % % % % % % % % % % % % % % % 
% % % % % % % % % % % % % % % % % % % % % % % % % % % % % % % % % % % % % % % % % % % % % % % % 
%                             #=#=# BEGIN PREAMBLE =#=#
% % % % % % % % % % % % % % % % % % % % % % % % % % % % % % % % % % % % % % % % % % % % % % % % 
% % % % % % % % % % % % % % % % % % % % % % % % % % % % % % % % % % % % % % % % % % % % % % % % 
% % % % % % % % % % % % % % % % % % % % % % % % % % % % % % % % % % % % % % % % % % % % % % % % 
% % % % % % % % % % % % % % % % % % % % % % % % % % % % % % % % % % % % % % % % % % % % % % % % 
% % % % % % % % % % % % % % % % % % % % % % % % % % % % % % % % % % % % % % % % % % % % % % % % 
% % % % % % % % % % % % % % % % % % % % % % % % % % % % % % % % % % % % % % % % % % % % % % % % 

% GRAPHICS / IMAGES
% - - - - - - - - - - - - - - - - - - - - - - - - - -
% The {graphicx} packages allows you to include images in your document.
\usepackage{graphicx}
% Set this if all graphics in one place: 
 \graphicspath{ {Images/} }
% The {placeins} package allow one to include the \FloatBarrier command,
% which prevents floats from passing it. The [section] option redefines
% of \section such that floats in a section stay in that section.
\usepackage[section]{placeins}
% - - - - - - - - - - - - - - - - - - - - - - - - - -


% MATH
% - - - - - - - - - - - - - - - - - - - - - - - - - -
% The {amsmath} package allows you to make nice looking equations.
\usepackage{amsmath}
% The {amsfonts} package gives you cool math symbols, like Z for the
% integers; {amsfonts} is loaded automatically when you load {amssymb}
\usepackage{amssymb}
% Extend 'mathbb':
\usepackage{mathbbol}
\newcommand{\Z}{\mathbb{Z}}
\newcommand{\R}{\mathbb{R}}
\newcommand{\RR}{\mathbb{R}^{2}}
\newcommand{\RRR}{\mathbb{R}^{3}}
\newcommand{\RN}{\mathbb{R}^{N}}
\newcommand{\Sph}{\mathbb{S}^{2}}
\newcommand{\so}{\mathbb{S}^{1}}
\newcommand{\soo}{\mathbb{S}^{2}}
\newcommand{\N}{\mathcal{N}}
\newcommand{\F}{\mathcal{F}}
\newcommand{\M}{\mathcal{M}}
\newcommand{\C}{\mathbb{C}}
\newcommand{\V}{\mathbb{V}}
\newcommand{\id}{\mathbb{1}} % requires package {mathbbol}
\newcommand{\oh}{\mathcal{O}}
% Vectors
\newcommand{\bvec}{\boldsymbol}
% - - - - - - - - - - - - - - - - - - - - - - - - - -


% LAYOUT
% - - - - - - - - - - - - - - - - - - - - - - - - - -
% The {geometry} package allows for margin customization.
% \usepackage{geometry}
% The {fancyhdr} package allows for header/footer customization.
% \usepackage{fancyhdr}
%\pagestyle{fancy}
% The {titlesec} packages allows for altering section/pages
% commands/styles.
%\usepackage{titlesec}
% -- Here, I am redefining a section break to mean that the next section
%  begins
%    on a new, clean page.
%\newcommand{\sectionbreak}{\clearpage}
% - - - - - - - - - - - - - - - - - - - - - - - - - -



% TABLES and FIGURES
% - - - - - - - - - - - - - - - - - - - - - - - - - -
% The {array} package is needed for extra control (might not be
% necessary)
%   \usepackage{array}
% The {multirow} package allows you to have more general tables
%  -- also allows to choose which cells get borders
%  e.g.:    
%  Year | Month |   Days w/ Data   |
%  --------------------------------|
%  2000 |  01   | 03, 04, 10       |
%       ---------------------------|
%       |  06   | 15, 16           |
%        --------------------------|
%       |  11   | 02, 23, 24, 27   |
%  --------------------------------
\usepackage{multirow}
% --btw, for floating tables, tabular environment must be placed 
%   inside a table environment; if you want your table to be 
%   exactly where you placed it, then only use tabular environment
% --the table environment also allows one to add a \caption{}, 
%   specify position (h: here, t: top, b: bottom, p: own page),
%   and use labels (e.g., \label{tab:myTable}) so it can be 
%   referenced later (\ref{tab:myTable})


% The {caption} and {subcaption} packages allow you to create
% multi-panel figures
\usepackage{caption}
\usepackage{subcaption}

% - - - - - - - - - - - - - - - - - - - - - - - - - -





% REFERENCES / BIBLIOGRAPHY
% - - - - - - - - - - - - - - - - - - - - - - - - - -
% The {biblatex} package is used for references
% Default style is numeric; can be changed to alphabetic or authoryear.
% Sorting: none (bib in order refs are cited); ynt (year, name title);
% etc
%\usepackage[style=authoryear,sorting=nyt]{biblatex}
%\addbibresource{references.bib}
% That said,
% for better author/year citing, the package {natbib} is often used
\usepackage[square]{natbib}
% More info: http://en.wikibooks.org/wiki/LaTeX/Bibliography_Management

% For INLINE BIBLIOGRAPHY entries:
% To have inline refs, must write: \nobibliography{<bib file>}
% If you still want bibliography at end, must still use
% \bibliography{..}, and also  \nobibliography*
% For inline reference:  \bibentry{<cite tag>}
\usepackage{bibentry}
% --- if you want to include a CV (e.g., in my dissertation proposal),
%  this is an ok way to do it, however, all CV items will also
%  reappear in the bibliography at the end...which you may or may not
%  care for (this package works by using the .bbl file that generates
%  the bibliograph).
% --- you might create a separate LaTeX project to do a CV, use
%  {bibentry} to get the proper bibliography formats, then copy and
%  paste
%
% - - - - - - - - - - - - - - - - - - - - - - - - - -


% Quick Reference:
% List environments: itemize, enumerate
% Math: equation, align, aligned, gather

% % % % % % % % % % % % % % % % % % % % % % % % % % % % % % % % % % % % % % % % % % % % % % % % 
% % % % % % % % % % % % % % % % % % % % % % % % % % % % % % % % % % % % % % % % % % % % % % % % 
% % % % % % % % % % % % % % % % % % % % % % % % % % % % % % % % % % % % % % % % % % % % % % % % 
% % % % % % % % % % % % % % % % % % % % % % % % % % % % % % % % % % % % % % % % % % % % % % % % 
% % % % % % % % % % % % % % % % % % % % % % % % % % % % % % % % % % % % % % % % % % % % % % % % 
% % % % % % % % % % % % % % % % % % % % % % % % % % % % % % % % % % % % % % % % % % % % % % % % 
%                              #=#=# END PREAMBLE  #=#=#
% % % % % % % % % % % % % % % % % % % % % % % % % % % % % % % % % % % % % % % % % % % % % % % % 

\begin{document}

\huge{Ionospheric Radio Wave Absorption}
\rule{\textwidth}{1pt}
\normalsize


\textbf{I. Radio Wave Propagation in the Ionosphere} \\  
  \\  
  \textbf{A. Basic Description} \\  
    A radio wave that passes through an ionized medium causes the
    electrons of that medium to vibrate about a mean position.
    If the vibrating electrons collide with the heavier, relatively
    stationary ions, radio wave energy is absorbed from the wave and 
    transferred to the medium. The rate of absorption depends on the 
    collision frequency (i.e., the number of collisions per oscillation) 
    -- that is, on $\nu/\omega$, where $\nu$ is the electron collision 
    frequency and $\omega$ is the angular wave frequency of the radio 
    wave.

    By IGY 1957-58, it was understood that high-latitude radio
    absorption is largely driven by geomagnetic disturbances, while mid-to-low
    latitude radio absorption is strongly local time dependent, or in
    other words, strongly controlled by the Sun's electromagnetic
    radiation.

  \textbf{B. Appleton-Hartree Quasi-Longitudianl Approximation} \\  
    For a wave propagating sufficiently close to the geomagnetic field
    direction, according to the classical Appleton-Hartree
    quasi-longitudinal approximation, the effect of the electron-ion
    collisions induced by the radio wave is to make the medium's index of
    refraction complex:

    \begin{gather}
      n^{2}=1-\frac{\omega^{2}_{N}/\omega^{2}}
      {1-i(\nu/\omega)\pm\omega_{H}cos(\theta/\omega)}=(\mu-i\chi)^{2}
    \end{gather}

    
    A complex index of refraction is indicative of wave
    absorption, where the real part indicates the phase velocity, while the
    imaginary part indicates the amount of absorption (energy loss) when the
    wave propagates through the medium. In this equation, $\mu$ and $\chi$ are the real 
    and imaginary parts of the refractive index, $\omega_{N}$ is the plasma 
    frequency, $\omega_{H}$ is the gyrofrequency, and $\theta$ is the angle 
    between the magnetic field and the radio wave's direction of propagation. 

    The wave amplitude of the radio wave, in the Appleton-Hartree
    approximation, varies with distance l like: 

    \begin{gather}
      exp\left[ -\frac{\omega\chi}{c}l\right] exp\left[
      i\omega t - \frac{\omega\mu}{c}l\right]
    \end{gather}

    The first term, governed by the complex part of the refractive index,
    is what controls the energy transfer from the wave to the medium
    (the ionospheric absorption), or in other words what controls the
    decay of the wave. The coefficient multiplying distance l is
    referred to as the "attenuation rate" K:

    \begin{gather}
      K=\frac{\omega}{c}\chi \quad\quad(in units of Np/m)
    \end{gather}

    NOTE: I think the unit "Np" here refers to the "Neper," which is a
    logarithmic unit for ratios of measurements of physical field and power
    quantities, such as gain and loss (but I'm not totally sure and should
    look into this more). 

    By plugging $\chi$ into the attenuation equation, one can see how the 
    attenuation of the radio wave depends on parameters such the electron density 
    and the collision frequency:
    \begin{gather}
      K=\frac{e^{2}}{2mc\epsilon_{0}}\cdot\frac{1}{\mu}\cdot\frac{N\nu}
      {\nu^{2}+(\omega\pm\omega_{H}cos(\theta))} 
    \end{gather}
    
    Here, e and m are electron charge and mass, c is the speed of light,
    and $\epsilon_{0}$ is the permittivity of free space.
    (\textbf{QUESTION}: is
    free space an assumption going into the Appleton-Hartree
    approximation?) If one sets $\mu=1$ for the lower ionosphere, one can compute the
    "total absorption" over a path for both E-mode (-) and O-mode radio
    waves (+):
    \begin{gather}
      A=4.6\cdot10^{-5}\int\frac{N\nu
      dl}{\nu^{2}+(\omega\pm\omega_{H}cos(\theta))^{2}} 
      \quad\quad (in units of dB)
    \end{gather}

    (\textbf{QUESTION}: why is $\mu=1$ is a good approximation?)
    
    Since the E-wave corresponds to a smaller denominator,
    $\omega-\omega_{H}cos(\theta)$, it is more readily absorbed by the
    ionosphere than the O-wave.
    
    Looking at the total absorption equation, one can see that at any frequency
    or height, the absorption increases linearly with electron density:
    more electrons in a given region of the ionosphere and you get more
    radio wave absorption. The height at which absorption is most sensitive 
    to changes in electron denity, N, is that height where 
    $\nu=\omega\pm\omega_{H}cos(\theta)$
   
    Although Appleton-Hartree quasi-longitudinal approximation is
    convenient analytically and has easily interpretable physics, it can
    ultimately be overly simplistic since it does not take into account 
    the dependence of on collision frequency, $\nu$, on electron energy.
    For example, the Appleton-Hartree approximation breaks down in the D
    region ionosphere (read: does not describe observations well when
    used to model the D region ionosphere).


  \textbf{C. Sen-Wyller Generalized Formulation} \\  
    As mentioned, sometimes the Appleton-Hartree approximation of radio
    wave absorption in the ionosphere can be overly simplistic. 
    For example, in the D region, the collision frequency is dependent on
    electron energy. One must generalize the Appleton-Hartree
    approximation by introducing the proper functional dependence of
    the collision frequency, $\nu$, on electron energy. 
    The Sen-Wyller formulation of radio wave absorption generalizes the
    Appleton-Hartree approximation in this way by introducing the 
    quantity $\nu_{m}$ in place of $\nu$, where $\nu_{m}$ is the collision
    frequency for a distribution of monoenergetic electrons and
    corresponds to the most probable velocity in the actual electron
    distribution. In the Appleton-Hartree approximation, $\nu$ corresponds to 
    the RMS velocity of the electron distribution. 
    
    One can show that $\nu$ is related to $\nu_{m}$. (Derive these.) At low altitudes, where $\nu\gg\omega$, the generalized Sen-Wyller formula
    recovers the Appleton-Hartree approximation by setting

    \begin{gather*}
      \nu=\frac{3}{2}\nu_{m}
    \end{gather*}

    At high altitudes, where $\nu\ll\omega$, the Appleton-Hartree formula
    is recovered from the generalized (Sen-Wyller) formula by setting

    \begin{gather*}
      \nu=\frac{5}{2}\nu_{m}
    \end{gather*}
    
    (\textbf{QUESTION}: Why 5/2 instead of 3/2?)

\textbf{II. Mid-to-Low Latitudes} \\  

\textbf{III. High Latitudes} \\  
   By IGY 1957-58, it was clear that high-latitude (auroral) radio
   absorption is driven by geophysical disturbances.

   There exist two major types of high-latitude ionospheric absorption 
   events: auroral absorption [AA] and polar cap absorption [PCA].
   The distinction between these two types of high-latitude absorption was
   originally made via characteristic time variations: although both can
   produce over 10dB of absorption on a 30-MHz riometer, PCA tends to be
   smooth and to last for several days, whereas AA is irregular and
   short-lived. Nowadays, with polar-orbiting spacecraft, the two types
   of high-latitude absorption events can further be distinguished by
   direct observations of particles during conjunctions with the
   ground-based riometers.

   The geographic distributions of PCA and AA differ as well: PCA
   covers the whole polar cap, whereas AA is confined to a belt near the
   auroral regions (the intensity of AA decreases both poleward and
   equatorward of this belt). Furthermore, AA occurs more frequently
   than does PCA.

   Hargreeves [1969] says that another type of absorption should have its
   own classifcation: Sudden Commencement Absorption [SCA], an enhancement
   associated with sudden-commencement magnetic storms.

   NOTE TO SELF: It seems like Hargreeves alternates between reserving
   the term ``auroral absorption'' for high-latitude absorption events
   in general and for the phenomenon specifically referred to as
   ``auroral absorption.'' This next passage is a great example; it very
   much appears to generaly describe both AA and PCA.

   So what is the physical mechanism of auroral absorption? According to
   Hargreeves [1969]:
   -- in short, the entry of energetic auroral particles cause radio waves
   to be absorbed by the upper atmosphere
   -- the stages may be represented like so: auroral particle
   precipitation (CAUSE) => production of ionization in the atmosphere =>
   equilibrium electron density (i.e., recombination processes) => radio
   absorption (EFFECT)
   -- so with a riometer, we measure absorption, but do so for example in
   order to infer particle precipitation (i.e., absorption is a
   quantitative observable that can be used to deduce information about
   other parameters not directly measurable, at least from the ground)

  \textbf{A. Auroral Absorption}
     From Hargreeves [1969] (decadal review of riometry):
     Auroral radio absorption observations were first reported Appleton 
     [1933] (Ionospheric investigations in high latitudes). 

     During periods of auroral and magnetic activity, the intensity of an
     ionospherically-reflected mid-to-high frequency radio wave is weakened,
     often to extinction.  Appleton [1933] inferred the cause of these radio
     wave extinctions to be due to absorption of the radio waves due to
     ``the production of ionization at abnormally low levels...below
     that at which we can detect ionization produced by ultra-violet
     light." The agent was thought to be "ionizing charged particles
     [which] produce electrification below the normal lower region
     [i.e., the E region]." 
     -- I'm not quite sure was this last bullet means; read more and figure 
     it out!!!

     Before IGY 1957-58, auroral absorption was studied by
     radio reflection methods (e.g., via "blackouts" recorded by ionosondes);
     Hargreeves [1969] says the reflection methods were too sensitive 
     -- "at frequencies of a
     few MHz...the amount of absorption...at high latitudes leads all too
     readily to the blackout condition, at which point the measurements cease
     to be quantitative." And so, much reflection work was based on blackout
     statistics.

     By the time of Hargreeves [1969], although reflection methods were still
     used, they had largely
     fallen out of favor as the "cosmic noise method" became popular,
     particularly the "riometer technique"


     The phenomenon called "auroral absorption" 
     is produced by the entry of auroral electrons 
     At high latitudes, Hargreeves [1969] says that AA is the type of
     absorption that occurs most often. AA is is a complex phenomenon
     (relative to similar phenomena, like PCA) and as of Hargreeves
     [1969] was not understood.

     AA is a sporadic phenomenon, growing and decaying with auroral and
     magnetic activity, yet doing so without any exact correspondence

     REVIEWS ON AURORAL ABSORPTION:
     1963: B. Hultqvist, "Studies of ionospheric absorption of radio waves by the
     cosmic noise method," in Radio Astronomical and Satellite Studies of the
     Atmosphere.
     1966: B. Hultqvist, "Ionospheric absorption of cosmic radio noise," in
     Space Science Reviews.
     1966: R. R. Brown, "Electron precipitation in the auroral zone," in
     Space Science Reviews.
     1967: C. G. Little, "Auroral absorption of radio waves," in Aurora and
     Airglow.


  \textbf{B. Polar Cap Absorption}
     Due to abnormal ionization produced by the incidence of solar protons 
     after an intense solar flare.
     Hargreeves [1969] says PCA is pretty well
     understood (solar energetic protons from solar flares)
     * for comprehensive review of Polar Cap Absorption, Hargreeves [1969]
     tells me to see Bailey [1964], "Polar Cap Absorption"

\end{document}




