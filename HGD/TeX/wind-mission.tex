% vim:tw=72 sw=2 ft=tex spell spelllang=en
% mainfile: ../hw1.tex
\textbf{Compare and contrast the orbits of ACE and WIND. What are the capabilities of both orbits?}

Prior to 2004, ACE was our dedicated solar wind monitor, whereas Wind's
orbit took it all over the place (see figures!). Depending on date, Wind
could be sampling near the magnetopause, in the magnetosheath, ahead of
Earth in the solar wind, or to the sides of Earth in the solar wind. In
conjunction with ACE during this time, one could have two nearby solar
wind data points for number density and velocity, which could be useful
in understanding how the solar wind varies across the solar wind /
magnetosphere interface and, ultimately, how that drives various
magnetospheric processes and phenomena.

After 2004, both spacecraft inhabited orbits about the L1 libration
point in the Sun-Earth system -- ACE in its Lissajous orbit, Wind in a
halo orbit (see last two figures of this section). This is good for data
comparisons and calibration between spacecraft, but we lose out on
having Wind measuring the state of the solar wind near the
magnetosphere's flanks. That said, Wind and ACE both orbiting L1 ensure
we have continuous ahead-of-Earth solar wind monitoring given one fails,
which is good for space weather forecasting.


\begin{figure}[h]
  \centering
  \begin{subfigure}[1a]{0.4\textwidth}
    \centering
    \includegraphics[width=\textwidth]{../images/Wind-2yrs.png}
  \end{subfigure}
  \hfill
  \begin{subfigure}[1b]{0.4\textwidth}
    \centering
    \includegraphics[width=\textwidth]{../images/ACE-2yrs.png}
  \end{subfigure}
  \caption{Orbits is GSE system. Left: Wind. Right: ACE.  (Feb 2003)}
  \label{fig-GSE}
\end{figure}

\begin{figure}[h]
  \centering
  \begin{subfigure}[1a]{0.4\textwidth}
    \centering
    \includegraphics[width=\textwidth]{../images/GSM-Wind.png}
  \end{subfigure}
  \hfill
  \begin{subfigure}[1b]{0.4\textwidth}
    \centering
    \includegraphics[width=\textwidth]{../images/GSM-ACE.png}
  \end{subfigure}
  \caption{Orbits is GSM system. Left: Wind. Right: ACE.  (Feb 2003)}
  \label{fig-gsm}
\end{figure}


\begin{figure}[h]
  \centering
  \begin{subfigure}[1a]{0.4\textwidth}
    \centering
    \includegraphics[width=\textwidth]{../images/GEI-J2000-Wind.png}
  \end{subfigure}
  \hfill
  \begin{subfigure}[1b]{0.4\textwidth}
    \centering
    \includegraphics[width=\textwidth]{../images/GEI-J2000-ACE.png}
  \end{subfigure}
  \caption{Orbits is GEI-J2000 system. Left: Wind. Right: ACE. (Feb
  2003)}
  \label{fig-gei}
\end{figure}


\begin{figure}[h]
  \centering
  \begin{subfigure}[1a]{0.4\textwidth}
    \centering
    \includegraphics[width=\textwidth]{../images/GSM-2006-07.png}
  \end{subfigure}
  \hfill
  \begin{subfigure}[1b]{0.4\textwidth}
    \centering
    \includegraphics[width=\textwidth]{../images/GSE-2006-07.png}
  \end{subfigure}
  \caption{Date: 2006-2007. Left: GSM. Right: GSE. Wind (pinkish), ACE
  (yellowish)}
  \label{fig-2006-07}
\end{figure}


http://sscweb.gsfc.nasa.gov/tipsod/
% vim:tw=72 sw=2 ft=tex spell spelllang=en
% mainfile: ../hw1.tex
\textbf{How is the WIND spacecraft powered?}
The Wind spacecraft is powered by surface-mounted solar arrays that
provide 370 W of power. The instruments require 144 W

\begin{figure}[ht]
  \centering
  \includegraphics[width=0.8\textwidth]{../images/Wind.jpg}
  \caption{\label{}}
\end{figure}.
