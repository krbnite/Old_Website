
Energy input from the solar wind into Earth's magnetosphere occurs
primarily at high latitdues -- or, perhaps more accurately, at high L
shell, which maps to high latitudes as one traces along the field to
lower altitudes.

Kinetic energy is deposited from the solar wind into the magnetosphere
through the precipitation of energetic particles, and electromagnetic
energy is dissipated via Joule heating, which for a current is defined
like:

\begin{gather}
  \vec{J}\cdot\vec{E} > 0
\end{gather}

For a single particle:

\begin{gather}
  q\vec{v}\cdot\vec{E} > 0
\end{gather}

Clearly then, Joule heating is when a charged particle is accelerated by the
electric field, resulting in a current density vector that has a parallel (as
opposed to an anti-parallel) component to the electric field. For
electrons, this equates to an anti-parallel velocity, and for positive
ions this is a parallel velocity.
In a gravitational field, the equivalent is letting a rock fall to the ground.




References:
Kelley, 1991: Poynting Flux Measurements on a Satellite: A diagnostic
tool for space research
