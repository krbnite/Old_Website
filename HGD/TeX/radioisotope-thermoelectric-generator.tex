% vim:tw=72 sw=2 ft=tex spell spelllang=en
% mainfile: ../hw1.tex
\textbf{Summarize how an RTG works for spacecraft power.}

There are various types of radioisotope thermoelectric generators. For
example, the Mars Science Laboratory uses what is called a Multi-Mission
RTG, which converts heat from radioisotope decay into electricity. The
MMRTG uses solid-state thermoelectric couples to convert the heat from
the decay of Plutonium-238 dioxide into electricity. The Mars Curiosity
Rover also utilizes the heat directly to maintain operating temperatures
for its payload. 

The MMRTG is a recent development in the RTG lineage. Older missions
such as the Ulysses mission utilized a design known as the General
Purpose Heat Source - RTG [GPHS-RTG]. This particular design was also
implemented on the Cassini, New Horizons, and Galileo missions. Weight
about 57 [kg], this RTG can generate about 300 [W] of electricity
(converted from about 4400 [W] of thermal power from the decay of
Plutonium-238). The conversion from heat to electricity was done using a
silicon-germanium thermoelectric unicouple.

In general, thermoelectric materials are those in which an electric
potential is generated by a temperature difference (Seebeck effect), or
vice versa (Peltier effect). There is also a phenomenon known as the
Thompson effect, which describes the heat generated by a current through
a conductor. 
