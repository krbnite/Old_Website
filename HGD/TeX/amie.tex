
empirical formual for specifying fac distributions
richmond and kamide [1988] developed an empirical approach to specify fac and
ionospheric potential distributions using ground- and space-based data; these
 empirical formulas were refined and developed, ultimately culminating in what
is known as assimilative mapping of ionospheric electrodynamics [amie] (cite:
richmond [1992]). amie inverts magnetic perturbations observe by mid-to-high
latitude ground stations to specify equivalent ionospheric currents. however,
these inversions are limited... raeder et al [2001] argued that translating
equivalent currents into realistic electric field estimates requires precise
knowledge about the distribution of ionospheric conductances. this concern can
be remedied a bit, though, since ionospheric conductances can be derived given
knowledge of solar ultraviolet radiances [wallis and budzinski, 1981] and the
fluxes of energetic particles precipitating from the magnetosphere [robinson
et al, 1987]. however, given an accurate description of ionospheric
conductances, there is still the issue of fukushima's theorem: that
perturbations produced by facs and poloidal closing currents exactly cancel on
the ground. this means that, inherently, amie is vulnerable to underestimating
fac intensities and ionospheric electric fields!



