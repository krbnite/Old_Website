% vim:tw=72 sw=2 ft=tex spell spelllang=en
% mainfile: ../hw1.tex
\textbf{SWEPAM data is often used to find the solar wind dynamic pressure. How is the dynamic pressure calculated?}

On the spot, one may compute the "dynamic pressure per unit mass" by multiplying the number density by the
square of the solar wind bulk velocity ($n V^{2}$).
If the composition of the solar wind is
well known, one can compute the actual dynamic pressure. Alternatively,
one can compute the, e.g., proton dynamic pressure ($m_{p} n V^{2} =
\rho V^{2}$). Since protons make
up most of the solar wind you might even go ahead and call that the
solar wind's dynamic pressure, especially in an off-the-cuff,
back-of-the-envelope type setting.

Pressure has units of force per area, which is \begin{math}
\frac{kg*m}{s^{2}}/m^{2} \end{math}. With some algebraic manipulation we
see that pressure is an
energy density 
\begin{math} \frac{kg*m^{2}}{s^{2}}/m^{3} \end{math}. This last set of
  units can be rewritten again to get the form we used above,
  $(\frac{kg}{m^{3}})(\frac{m^{2}}{s^{2}})$, which tells us that dynamic
  pressure is some kind of "kinetic energy density" of the solar wind. It should be noted that another possible definition of dynamic pressure
may be found in the literature, although not often used in space
physics, $\frac{1}{2} \rho V^{2}$, which is exactly how you'd expect a
kinetic energy density to look.


SWEPAM provides the bulk solar wind observations for the Advanced
Composition Explorer, recording properties of the solar wind such
as number density, the bulk flow velocity, and temperature. \citep{McComas1998}.
SWEPAM measures both electrons and ions. The instrumentation has Ulysses
heritage (literally Ulysses spare parts), although refinements were
made. 

You can check out some real-time solar wind dynamic pressure
observations at http://www.swpc.noaa.gov/SWN/. 
