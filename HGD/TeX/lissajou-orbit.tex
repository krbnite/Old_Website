% vim:tw=72 sw=2 ft=tex spell spelllang=en
% mainfile: ../hw1.tex
\textbf{What is a Lissajous orbit, as pertaining to the ACE spacecraft orbiting the L1 point?}

Lissajous orbits are a type of non-periodic (or, sometimes referred to
as quasi-periodic) orbits about a libration point in a three-body system
(i.e., about a point where all gravitational forces cancel). They can
look fairly messy or nonintuitive on an orbital diagram, yet they are
natural orbits that require minimal energy for station-keeping (i.e.,
minimal cost to keep the spacecraft in the orbit via subtle propulsion
maneuvers). 


\begin{figure}[h]
  \centering
  \includegraphics[width=0.7\textwidth]{../images/lissajous.png}
  \caption{Lissajous orbit of the ACE spacecraft. Image Credit:
  http://www.srl.caltech.edu/ACE/images/lissajous.gif}
  \label{fig-lissajous}
\end{figure}

The Advanced Composition Explorer [ACE] \citep{Stone1998} is in a
lissajous orbit. Orbit maneuvers are used to keep ACE in the lissajous
orbit --- about 3 [lbm/year] of fuel is required for this. It is not
enough though to simply maintain its Lissajous orbit: ACE must also maintain proper orientation.  To maintain orientation of the spacecraft's high-gain antenna
(HGA), ACE must perform attitude adjustments; this requires about double
the fuel it takes to maintain orbit.
