% mainfile: ../hw4.tex
\textbf{Briefly discuss the roles of whistlers, hiss, and chorus in the
inner magnetosphere.}

Plasmaspheric hiss plays a role in controlling the two-zone structure of
the radiation belts by sculpting the slot region between the inner and
outer belts. Whistler-mode chorus evolves into plasmaspheric hiss. The
interaction of whistler-mode waves with radiation belt particles can
scatter the particles into the loss cone (i.e., affect their trajectory
such that they are lost to the atmosphere as they approach their mirror
point); such scattering de-populates the radiation belts and the ring
current. 

Cyclotron resonance can occur when a particle's motion is such that an
oncoming whistler-mode wave is Doppler-shifted such that it appears to have a
frequency matching the particle's cycltron frequency, or one of its
harmonics. An example of this cyclotron resonance is Landau damping;
in a collisionless plasma, this type of mechanism takes on the role that 
particle-particle collisions provide in a collision-dominated
gas or plasma, i.e., such wave-particle interactions allow for dissipation.
