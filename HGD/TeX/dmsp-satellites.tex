
DMSP satellite are 3-axis stabilized spacecraft that fly in circulat,
sun-synchronous, polar (inclination $98.7^{\circ}$) orbits at an
altitude of $\sim840$ km. The geographic local times of the orbits are
near the 1800/0600 (F11, F13) or the 2100/0900 (F12, F14, F15) local
time meridians. The offset between the geographic and geomagnetic poles
means that the DMSP fleet samples a wide range of magnetic local times
over the course of a day. 

* the ascending nodes of DMSP orbits are on the dusk side of Earth; this
means that in the evening sector, the DMSP satellites move northwest; in
the morning sector, the satellites move southeast
 -- mental help: looking down on the Earth's north geographic pole, west is clockwise
 and east is counter clockwise

 * each DMSP satellite carries a suite of instruments to measure (1) fluxes
 of auroral particles (Special Sensor J4 [SSJ4]) and (2) densities,
 temperatures, and drift motions of ionospheric ions and electrons
 (Special Sensor for Ions, Electrons, and Scintilations [SSIES]). Later
 DMSP spacecraft (ge F12, I think) carry magnetometers as well (Special
 Sensor for Magnetic Fields [SSM])


The SSJ4 Sensors
* mounted on the top surfaces of DMSP satellites to measure fluxes of
down-coming electrons and ions in 19 logarithmically spaced energy steps
between approximately 30 eV and 30 keV (see Hardy et al, 1984)
* a full spectrum is compiled every second
* on F15, fluxes of ions with energy below 1 keV are unavailable

The SSIES Sensors
* consists of spherical Langmuir probes mounted on 0.8 m booms to
measure densities and temperatures of ambient electrons
* has three different sensors mounted on a conducting plate facing the
ram direction: (1) the ion traps, which measure the total ion densities;
(2) ion drift meters, which measure horizontal ($V_{H}$) and vertical
($V_{V}$) cross-track components of plasma drifts; and (3) retarding potential 
analyzers [RPAs], which measure ion temperatures and in-track components of 
plasma drift ($V_{\par}$)
* see Rich and Hairston, 1994
* shapes of RPA current-voltage sweeps can be used to determine the
masses of contributing ion species
* b/c of relatively small uncertainties about spacecraft potentials, RPA
estimates of $V_{\par}$ often have large error bars
* the drift meters measure $V_{H}$ at 6 Hz
* F11 has a noisy tape recorder, which limits the usefulness of it data

SSM Sensors
* triaxial fluxgate magnetometers are mounted on the bodies of the F12-F14
spacecraft
* the mag on F15 is mounted on a 5-m boom, reducing its susceptibility
to spacecraft-generated contamination
* magnetic vectors are samples at 12 Hz
* 1-second averages are calculated as
$\delta\vec{B}=\vec{B}_{meas}-\vec{B}_{IGRF}$
* data are presented as $\delta{B_{x}}$, $\delta{B_{y}}$, and
$\delta{B}_{z}$ in spacecraft-centered coordinates, where the X- and
Y-axes point toward spacecraft nadir and along the velocity vector,
respectively; the Z-axis compltes the right-hand system


Analyzing DMSP data
* two governing equations for analyzing combined SSM and SSIES data are
Ampere's law and current continuity:
(i) $j_{\par}=\frac{1}{\mu_{0}}
\left[\nabla\times\delta\vec{B}\right]_{\par}$
(ii) $\frac{\partial{j_{\par}}}{\partial{s}} =
-\nabla_{\perp}\vdot\vec{j}_{\perp}$
-- s represents an infinitesimal distance along the field line
* in the ionosphere, the current density perpendicular to the Earth's
magnetic field $\vec{B}_{E}$ is
$\vec{j}_{perp}=\sigma_{P}\vec{E}-\sigma_{H}(\vec{E}\times\vec{b})$
-- where the sigmas are the Pedersen and Hall conductivities, $\vec{b}$
is a unit vector along $\vec{B}_{E}$
* we can integrate the current continuity equation, (ii), from the
satellite location to the bottom of the ionosphere (along the field
line, I think) to get:
$j_{\par}=\nabla_{\perp}\vdot\vec{I}_{\perp} =
\nabla_{\perp}\vdot\left[\Sigma_{P}\vec{E}-\Sigma_{H}(\vec{E}\times\vec{b})\right]$
-- the capital sigmas are the Pedersen and Hall conductances

