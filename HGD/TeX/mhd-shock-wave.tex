% mainfile: ../hw3.tex
\textbf{For MHD shock waves, why is $\left[\vec{u_{t}}\right] ||
\left[\vec{B_{t}}\right]$?}

I'm not sure if this is true for all frames, but for the
deHoffman-Teller frame it is.

The dHT frame is defined such that $\left| \vec{u_{1}} \times
\vec{B_{1}}\right| =
0$. From the Rankine-Hugoniot relations for MHD, we can then immediately
write down $\left| \vec{u_{2}} \times \vec{B_{2}}\right| = 0$.

By property of the cross product, what we've just said is that in the
dHT frame, the magnetic field and the plasma flow are parallel to each
other in both regions delimited by the shock --- that is, the upstream
magnetic field is parallel to the upstream flow and the downstream
magnetic field is parallel to the downstream flow.

Since the two quantities are parallel in both domains they are found in, we
need only consider the unit vectors, $\vec{d_{1}}$ and $\vec{d_{2}}$,
associated with the directions of these quantities in each domain.

From here, it is obvious that the relationship $\left[\vec{u_{t}}\right] ||
\left[\vec{B_{t}}\right]$ holds. Simply rewrite each quantity in terms
of it's (unitless) unit vector, $\vec{d}$.
\begin{center}
\begin{math}
  \left[ \vec{B} \right] = \left[ \vec{B_{t}} + \vec{B_{n}} \right] ||
  \left[ \vec{d_{t}} + \vec{d_{n}} \right] || \left[
  \vec{u_{t}}+\vec{u_{n}} \right] = \left[ \vec{u} \right]
\end{math}
\end{center}

The relationship is hidden in there, but it's there:
\begin{center}
  \begin{math}
    \left[ \vec{B_{t}} \right] || \left[ \vec{d_{t}} \right] || \left[
    \vec{u_{t}} \right]
  \end{math}
\end{center}





% mainfile: ../hw3.tex
\textbf{For a MHD parallel shock, show that it reduces to a hydrodynamic
shock.}

By "parallel" shock, we mean that the upstream flow is parallel to the
upstream magnetic field, and that the downstream flow is parallel to the
downstream magnetic field.

Let's start by writing down these quantities:
\begin{center}
  \begin{align}
    \vec{u}_{up} = \left< u_{up}, 0, 0 \right> \\
    \vec{u}_{down} = \left< u_{down}, 0, 0 \right>\\
    \vec{B}_{up} = \left< B_{up}, 0, 0 \right>\\ 
    \vec{B}_{down} = \left< B_{down}, 0, 0 \right> 
  \end{align}
\end{center}

Then we write down the Rankine-Hugoniot relations:
\begin{align}
  \left[ B_{x}\right]^{2} = 0 \\
  \left[ u_{x}B_{y} - u_{y}B_{x}\right] = 0 \\
  \left[ \rho u_{x}\right] = 0 \\
  \left[ \rho u_{x}^{2}+P+B_{y}^{2}/2\mu_{0}\right] = 0 \\
  \left[ \rho u_{x}u_{y}-B_{x}B_{y}/\mu_{0}\right] = 0 \\
  \left[ \frac{1}{2}\rho
  u^{2}u_{x}+\frac{\Gamma}{\Gamma-1}pu_{x}+\frac{ B_{y}(
  u_{x}B_{y}-u_{y}B_{x} )}{\mu_{0}}\right] = 0
\end{align}

Given equations (1)-(4), these relations reduce to:
\begin{align}
  \left[ B_{x}\right]^{2} = 0 \\
  \left[ \rho u_{x}\right] = 0 \\
  \left[ \rho u_{x}^{2}+P \right] = 0 \\
  \left[ \frac{1}{2}\rho
  u^{3}+\frac{\Gamma}{\Gamma-1}pu_{x}\right] = 0
\end{align}

The most obvious thing, I think, about these equations is that the
magnetic field is uncouple from the plasma flow. That is, setting up the
equations for the shock is independent of the magnetic field -- so a
parallel shock in the MHD setting does not only reduce to a hydrodynamic
shock in some limit, but is indeed a hydrodynamic shock with or without
the presence of a magnetic field.




% mainfile: ../hw3.tex
\textbf{Discuss slow shocks found in nature; give an example.}

In interplanetary space, one may detect a slow shock in the solar wind given the right
instruments. In general, as the spacecraft passes through a shock, the
temperature, density, and speed will be observed to increase.

The difference between a fast and a slow shock in these types of
observations is the magnetic field strength: given the above
observations of temperature, density, and speed, if the field strength is observed to
increase, it is a fast shock; if observed to decrease, it is a slow
shock.

So when do we see slow shocks in the solar wind? Near corotaing
interaction regions, of course! 
