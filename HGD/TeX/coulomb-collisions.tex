% mainfile: ../hw4.tex
\textbf{Compare the loss rates of protons, He+ ions, and O+ ions due to
charge exchange and coulomb collision between 0.1-300 keV. Why can it be
said that He+/H+ and O+/H+ ratios are good proxies for He+ and O+
lifetimes above ~50 keV?}


From the last slide of lecture 8, we see that for both He+ and O+, the
lifetime of charge exchange steadily decreases until about 100 keV,
where both begin increasing --- He+ rapidly so, and O+ gradually. O+
participates in charge exchange in the range of 0.1-1.0 keV, while He+
does not; both partcipate in charge exchange between 1.0-300 keV. As for
Coulomb collisions, according the the last slide of lecture 8, both O+
and He+ begin experiencing this type of interaction around 1 keV; the
lifetime of this interaction steadily increases for both species as the
energy increases to 300 keV.

\begin{figure}[ht]
  \centering
  \includegraphics[width=0.5\textwidth]{../images/lec8_lastslide.png}
  \label{fig1}
\end{figure}

From some figures in \citet{ebihara1998}, we see that the Coulomb lifetimes of
protons above ~50 keV is almost an order of magnitude longer for protons
than the heavier ions, and that the the charge exchange lifetime for
protons for energies above 50 keV is always at least one order of
magnitude longer, if not more. This tells us that the protons
populations at these energies are much more stable than for the heavier
ions. 

\begin{figure}[ht]
  \centering
  \includegraphics[width=0.5\textwidth]{../images/ebihara1.png}
  \label{fig2}
\end{figure}
\begin{figure}[ht]
  \centering
  \includegraphics[width=0.7\textwidth]{../images/ebihara2.png}
  \label{fig3}
\end{figure}

