
* due to abnormal ionization produced by the incidence of solar protons 
after an intense solar flare; the related phenomenon called "auroral absorption" 
is produced by the entry of auroral electrons (these make up the two
major types of high-latitude ionospheric absorption events)
* the distinction between these two types of auroral absorption was
originally made via characteristic time variations: although both can
produce over 10dB of absorption on a 30-MHz riometer, PCA tends to be
smooth and to last for several days, whereas AA is irregular and
short-lived; with polar-orbiting spacecraft, identification is aided by
direct observations of particles
* the geographic distributions of PCA and AA tend to differ as well: PCA
covers the whole polar cap, whereas AA is confined to a belt near the
auroral regions (the intensity of AA decreases both poleward and
equatorward of this belt)
* Hargreeves [1969] says that another type of absorption should have its
own classifcation: Sudden Commencement Absorption [SCA], an enhancement
associated with sudden-commencement magnetic storms
* for comprehensive review of Polar Cap Absorption, Hargreeves [1969]
tells me to see Bailey [1964], "Polar Cap Absorption"
* at high latitudes, Hargreeves [1969] says that AA is the type of
absorption that occurs most often; he says whereas PCA is pretty well
understood (solar energetic protons from solar flares), AA is much more complex
and not understood  (for more on AA, see AA entry)



REFERENCES
Hargreaves et al 1969: ..
Baily 1964: Polar Cap Absorption
