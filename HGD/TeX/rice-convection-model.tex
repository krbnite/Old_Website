
This info is from Cheryl's 2004 paper:
* as far as FACs are concerned, the RCM implements Vasyliunas'
prescription --- that FACs are the consequence of misalignments between
plasma pressure gradients and flux tube volume gradients
* RCM only simulates the dynamics of the inner magnetosphere; it treats
R1 FACs and the coupling to interplanetary space as outside it region of
scope 
* RCM assumes that plasma compresses adiabatically while calculating the
growth of the ring current, the formation of R2 FACs, and consequent
redistributions of electric fields in the ionosphere and inner
magnetosphere
* Caveat: Erickson and Wolf [1980] pointed out that adiabatic
compression (what RCM uses) leads to insupportably high plasma pressures. 
Kivelson and Spence [1988] suggested that ion loss via gradient-curvature 
drift to the magnetopause lowers plasma pressure. Borovsky et al [1998]
empirically demonstrated that significant departure from adiabatic
transport occurs at radial distances between 6.6-15 RE in the
magnetotail (they concluded that ion precipitation is responsible for a
large fraction of mass loss)
* Performance: using RCM,  Garner et al [2004] reproduced electric field
distributions observed by the Combined Release and Radiaton Effects
Satellite [CRRES] in the inner magnetospher for the June 1991 storm;
Burke et al [1998] show that RCM reproduced electric field distributions 
observed by DMSP satellites in the ionosphere for the same storm.
However, Garner et al [2004] also showed that RCM consistently predicted
more plasma pressure than was actually measured in the ring current
population 
