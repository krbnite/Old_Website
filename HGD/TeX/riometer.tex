
Single-Frequency Riometers
Multi-Frequency Riometers
Swept-Frequency Riometers
Stepped-Frequency Riometers


* in general, the riometer is a radio technique and 
offers distinct advantages as a monitoring and synoptic
tool: one can make continuous observations,  it is equally 
effective day or night, it is not obscured by cloud cover; furthermore,
a riometer is a fairly simple instrument and if treated properly, gives
reliable measurements (Hargreeves [1969])
* for auroral studies, a riometer can offer a long, uninterrupted,
continuous data set


\textbf{The Cosmic Noise Method}
1. The apparent intensity of the cosmic radio emission is monitored
continuously on a stable receiver.
2. The galatic radio flux is contant over long periods of time, so
presumably any changes in the apparent intensity from one day to the
next at the same sidereal time represent corresponding variations of
ionospheric absorption.
3. Since this method depends on wave propagation through the ionosphere,
the frequency must be comfortably above f0F2. In the mid-latitudes, the
amount of absorption at these frequencies is small and varies slowly throughout the day (it
is "solar controlled"); given that there often exists ``receiver
drift,'' it is fairly tough to parse out what the cosmic-noise intensity
is, versus the drift, versus ionospheric absorption. At high latitudes,
however, this is not the case: the absorption is strong and structured.
This allows one to determine the background level (often called the
"quiet-day curve"). 

\textbf{The Riometer}
1. The riometer achieves high gain stability by switching rapidly between
the antenna and a local noise source.
2. The local noise source is continuously adjusted so that its power output
equals that received by the antenna.
3. Thus the receiver acts as a sensitive null indicator, in which gain
variations are unimportant.
4. Ultimately, a recording is made of the current through the noise
source, the current being linearly related to the power output.

As of 1969, Hargreaves had reported that there existed a need to
``simplify the data processing by which the negative deflection on a
chart that is nonlinear in decibels is converted to a
linear scale of decibels... [A] means of removing the quiet-day curve at the
instrument and of producing on the spot a record [that is] linear in
absorption against time  would be [AWESOME!]''
* what is the current state of this?





SEE: Galactic Radio Flux
SEE: Receiver
SEE: Sidereal Time
SEE: f0F2
SEE: Quiet-Day Curve
SEE: Cosmic Noise
SEE: Null indicator
SEE: Gain (gain stability of receiver)

