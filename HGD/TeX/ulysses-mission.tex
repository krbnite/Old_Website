% mainfile: ../hw1.tex
\textbf{Discuss the finding from Ulysses regarding the outflow of the solar wind over the southern pole as compared to the northern pole. Cite references.}

Prior to the Ulysses mission, the solar wind speed (on average) was
expected to increase steadily with latitude. The Ulysses mission showed
that, in general, this is not true: the increase in speed between the
ecliptic region and between the polar region is almost step-function
like [see \citet{Ulysses2009} website reference] (e.g., figure \ref{fig1}).

\begin{figure}[h]
  \centering
  \begin{subfigure}[1a]{0.3\textwidth}
    \centering
    \includegraphics[width=\textwidth]{../images/Ulysses1.png}
  \end{subfigure}
  \hfill
  \begin{subfigure}[1b]{0.6\textwidth}
    \centering
    \includegraphics[width=\textwidth]{../images/Ulysses2.jpg}
  \end{subfigure}
  \caption{Solar wind speed does not gradually increase in latitude, but instead ramps up in a step-function like fashion}
  \label{fig1}
\end{figure}

Before the Ulysses mission, we had measurements made only within the
elcliptic plane; this allowed for only a basic 2D understanding of the
solar wind structure. Ulysses was the first and only spacecraft to
ascend into heliographic latitudes outside the ecliptic plane, rendering
observations from Ulysses extremely important to our understanding of
the heliosphere.

One such set of observations are those concerning north-south
assymetries. Initially, the Ulysses mission was supposed to be a
dual-spacecraft mission, which would have greatly benefitted questions
concerning assymetries between hemispheres \citep{Marsden1996}. However, the
mission was not unsuccessful without a second spacecraft: the
pole-to-pole scan prior to 1996 was rapid enough, with conditions
constant enough, to meaningfully compare the characteristics of each
hemisphere. Such comparisons showed numerous modest assymetries,
e.g., \citet{Goldstein1996}, who showed---using SWOOPS plasma
experiment---that the velocity is on average 13-24 [km/s] greater in
the northern polar region than in the south; they also showed that the
temperature in the north is higher, and that this cannot be explained
by velocity assymetry and the dependence of solar wind temperature on
speed. Furthermore, they showed by comparisons with calculations based
on models and magnetograph data that the expansion of open magnetic
flux was greater in the south rather than the north.

The low-energy particle measurements made by the HI-SCALE instrument
described in \cite{Lanzerotti1996}
also show modest north-south assymetries. In this paper, flux variations
with a period of $\approx26$ days were observed in the southern, but not
the northern hemisphere. Additionally, solar particle events with
near-equatorial origins were observed at high southern latitudes, but
again not in the north. The flux of low-energy anomalous oxygen was
found to be almost 50\% higher in the north, while the iron flux was
observed to be $\approx2$ times larger in the south.
