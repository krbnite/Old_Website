% mainfile: ../hw4.tex
\textbf{ What are the defining characteristics of the plasmasphere, ring
current, and plasma sheet?}

The plasmasphere's plasma is referred to as "cold" -- that is, it
largely consists of low energy particles in 1-10 eV range. It is a
relatively dense plasma compared to other magnetosperic regions, having
number density between 100-1000 per cubic centimeter. It is composed
primarily of protons (~77\%), but also contains a decent amount of
helium ions (~20\%), and a small amount of oxygen ions (~3\%). In
spacecraft data, the plasmasphere is delineated by a sharp, obvious
boundary called the plasmapause. Due to the high density, one can
consider the plasmasphere the region of the inner magnetosphere
holding most of the mass.

The ring current is "hotter" than the plasmapause, housing particles
with energies between 1-400 keV. Having such particles that are so much
more energetic than the plasmasphere, one might refer to the ring
current as the region of the inner magnetosphere holding the most
energy. That said, as far as mass goes -- this is a tenuous plasma,
measuring only 1-10 particles per cubic centimeter. In this region, a
spacecraft will measure electrons and ions moving in opposite directions
(thus a current). 

The plasma sheet is located in the tailward region of the magnetosphere
on the stretched field lines. This is a rarefied plasma having number
densities between 0.1-1.0 per cubic centimeter. The plasma population is
largely composed of hydrogen and oxygen ions and holds a fairly
invariant relationship between the ion and electron temperatures: $T_{i}
= 7T_{e}$.


