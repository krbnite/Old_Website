% mainfile: ../hw3.tex
\textbf{Summarize the difference between the ion foreshock and the
electron foreshock.}

The foreshock is the region adjacent to the bow shock, outside of
Earth's magnetosphere, populated by energized charged particles that
have bounced off the bow shock instead of being transmitted; also found
in this region are magnetosheath particles that have somehow propagated
back across the bow shock.
The solar wind electrons that bounce off the bow shock, with speeds greater than that of the ion
populations, are able to propagate further sunward than the ions and,
so, the foreshock has an upstream region known as the electron
foreshock;  the downstream region, dominated by ion populations, is called the ion foreshock. 
