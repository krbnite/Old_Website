
* Winningham [1975] distinguished between two types of auroral electron
precipitation: boundary plasma sheet [BPS] precipitation and central
plasma sheet [CPS] precipitation
 -- BPS precipitation occurs in the poleward part of the nigthsdie oval
 and is highly structured in energy-versus-time spectrograms

* DeCoster and Frank [1979] employed similar nomenclature to that of
Winningham [1975] when describing observed particle populations in 
the magnetosphere: the "plasma sheet boundary layer" and
the "central plasma sheet"
 -- the central plasma sheet precipitation coincides with Winningham
 [1975]
 -- the PSBL is a region of intense electric fields and
 velocity-dispersed ion fluxes; it does not exactly correspond to the BPS 
 precipitation described above since it covers a much narrower range of 
 magnetic latitudes than the BPS.
