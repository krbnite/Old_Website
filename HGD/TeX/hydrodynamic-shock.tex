% mainfile: ../hw3.tex
\textbf{For a MHD parallel shock, show that it reduces to a hydrodynamic
shock.}

By "parallel" shock, we mean that the upstream flow is parallel to the
upstream magnetic field, and that the downstream flow is parallel to the
downstream magnetic field.

Let's start by writing down these quantities:
\begin{center}
  \begin{align}
    \vec{u}_{up} = \left< u_{up}, 0, 0 \right> \\
    \vec{u}_{down} = \left< u_{down}, 0, 0 \right>\\
    \vec{B}_{up} = \left< B_{up}, 0, 0 \right>\\ 
    \vec{B}_{down} = \left< B_{down}, 0, 0 \right> 
  \end{align}
\end{center}

Then we write down the Rankine-Hugoniot relations:
\begin{align}
  \left[ B_{x}\right]^{2} = 0 \\
  \left[ u_{x}B_{y} - u_{y}B_{x}\right] = 0 \\
  \left[ \rho u_{x}\right] = 0 \\
  \left[ \rho u_{x}^{2}+P+B_{y}^{2}/2\mu_{0}\right] = 0 \\
  \left[ \rho u_{x}u_{y}-B_{x}B_{y}/\mu_{0}\right] = 0 \\
  \left[ \frac{1}{2}\rho
  u^{2}u_{x}+\frac{\Gamma}{\Gamma-1}pu_{x}+\frac{ B_{y}(
  u_{x}B_{y}-u_{y}B_{x} )}{\mu_{0}}\right] = 0
\end{align}

Given equations (1)-(4), these relations reduce to:
\begin{align}
  \left[ B_{x}\right]^{2} = 0 \\
  \left[ \rho u_{x}\right] = 0 \\
  \left[ \rho u_{x}^{2}+P \right] = 0 \\
  \left[ \frac{1}{2}\rho
  u^{3}+\frac{\Gamma}{\Gamma-1}pu_{x}\right] = 0
\end{align}

The most obvious thing, I think, about these equations is that the
magnetic field is uncouple from the plasma flow. That is, setting up the
equations for the shock is independent of the magnetic field -- so a
parallel shock in the MHD setting does not only reduce to a hydrodynamic
shock in some limit, but is indeed a hydrodynamic shock with or without
the presence of a magnetic field.
