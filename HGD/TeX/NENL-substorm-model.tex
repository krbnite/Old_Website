% mainfile: ../hw5.tex
\textbf{Summarize the basic concepts of the NENL model. How is such a
model applied in solar physics?}

The initial motivation for the NENL model came about when it was
recognized that the auroral activity associated with substorms manifests
in association with southward turnings of the interplanetary magnetic
field. 

Reconnection on the dayside between the Earth's magnetosphere and the
interplanetary magnetic field creates "open" flux, which is dragged back
into the tail lobes, where it is stored. This process continues until
the substorm onset is triggered. During the substorm expansion phase,
the previously stored magnetic energy in the tail lobes is converted
into thermal and kinetic energy via reconnection, heating the associated plasma and
causing it to accelerate away from the reconnection site, both Earthward
and tailward.

From McPherrons's "Development of the Near-Earth Neutral Line Model:
"Plasma flowing Earthward piles up in the inner region and is then
diverted towards the flanks. Precipitation of this heated and turbulent
plasma creates an auroral disturbance that propagates poleward as the
pileup moves tailward. At the same time the pressure gradients and
vorticity of the plasma flow produce the substorm current wedge. The
outward current in the current wedge accelerates electrons into the
ionosphere generating bright discrete aurora in the westward surge. When
the pileup of plasma and flux approaches the location of the neutral
line the neutral line begins to move tailward. This initiates the
recovery phase of the substorm with a global expansion of the plasma
sheet and a broadening of the auroral oval over most of the night side."
