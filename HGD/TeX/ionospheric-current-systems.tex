
Notes from email from Cheryl:
* during any kind of geomagnetic activity, the equator current 
systems are generally quiet and static compared to the high-latitude 
current systems, where there are FACs

* polar FACs: region 0, 1, and 2 FACs (although in the magnetosphere,
these current systems map equatorially and all over the place, in the
ionosphere, they are polar currents)

* if there are polar cap current systems, they have to close somewhere;
do they close in the region 0 current system? Cheryl says, if so, then
there must be horizontal currents at F-region altitudes (i.e., Pedersen
currents, which are parallel to the electric field and perpendicular the
magnetic field); her argument is such: polar cap conductivities are low
given that the electron preciptitaion in the polar cap is soft (low
energy); if Pedersen current exists and Pedersen conductivity is low,
then by the continutiy of J, you get a huge electric field
(J=conductivity*electricField); so there must be large polar cap
electric field (which is what she thinks Fig1 in Mishin [2013] is
showing)






REFERENCES
* Huang, Cheryl. Personal Communication.
* Mishin et al [2013]: Short-Circuit in the Magnetosphere-Ionosphere
Electric Circuit
