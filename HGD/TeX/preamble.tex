\documentclass[12pt,a4paper]{article}

% % % % % % % % % % % % % % % % % % % % % % % % % % % % % % % % % % % % % % % % % % % % % % % % 
% % % % % % % % % % % % % % % % % % % % % % % % % % % % % % % % % % % % % % % % % % % % % % % % 
% % % % % % % % % % % % % % % % % % % % % % % % % % % % % % % % % % % % % % % % % % % % % % % % 
% % % % % % % % % % % % % % % % % % % % % % % % % % % % % % % % % % % % % % % % % % % % % % % % 
% % % % % % % % % % % % % % % % % % % % % % % % % % % % % % % % % % % % % % % % % % % % % % % % 
% % % % % % % % % % % % % % % % % % % % % % % % % % % % % % % % % % % % % % % % % % % % % % % % 
%                             #=#=# BEGIN PREAMBLE =#=#
% % % % % % % % % % % % % % % % % % % % % % % % % % % % % % % % % % % % % % % % % % % % % % % % 
% % % % % % % % % % % % % % % % % % % % % % % % % % % % % % % % % % % % % % % % % % % % % % % % 
% % % % % % % % % % % % % % % % % % % % % % % % % % % % % % % % % % % % % % % % % % % % % % % % 
% % % % % % % % % % % % % % % % % % % % % % % % % % % % % % % % % % % % % % % % % % % % % % % % 
% % % % % % % % % % % % % % % % % % % % % % % % % % % % % % % % % % % % % % % % % % % % % % % % 
% % % % % % % % % % % % % % % % % % % % % % % % % % % % % % % % % % % % % % % % % % % % % % % % 

% GRAPHICS / IMAGES
% - - - - - - - - - - - - - - - - - - - - - - - - - -
% The {graphicx} packages allows you to include images in your document.
\usepackage{graphicx}
% Set this if all graphics in one place: 
 \graphicspath{ {Images/} }
% The {placeins} package allow one to include the \FloatBarrier command,
% which prevents floats from passing it. The [section] option redefines
% of \section such that floats in a section stay in that section.
\usepackage[section]{placeins}
% - - - - - - - - - - - - - - - - - - - - - - - - - -


% MATH
% - - - - - - - - - - - - - - - - - - - - - - - - - -
% The {amsmath} package allows you to make nice looking equations.
\usepackage{amsmath}
% The {amsfonts} package gives you cool math symbols, like Z for the
% integers; {amsfonts} is loaded automatically when you load {amssymb}
\usepackage{amssymb}
% Extend 'mathbb':
\usepackage{mathbbol}
\newcommand{\Z}{\mathbb{Z}}
\newcommand{\R}{\mathbb{R}}
\newcommand{\RR}{\mathbb{R}^{2}}
\newcommand{\RRR}{\mathbb{R}^{3}}
\newcommand{\RN}{\mathbb{R}^{N}}
\newcommand{\Sph}{\mathbb{S}^{2}}
\newcommand{\so}{\mathbb{S}^{1}}
\newcommand{\soo}{\mathbb{S}^{2}}
\newcommand{\N}{\mathcal{N}}
\newcommand{\F}{\mathcal{F}}
\newcommand{\M}{\mathcal{M}}
\newcommand{\C}{\mathbb{C}}
\newcommand{\V}{\mathbb{V}}
\newcommand{\id}{\mathbb{1}} % requires package {mathbbol}
\newcommand{\oh}{\mathcal{O}}
% Vectors
\newcommand{\bvec}{\boldsymbol}
% - - - - - - - - - - - - - - - - - - - - - - - - - -


% LAYOUT
% - - - - - - - - - - - - - - - - - - - - - - - - - -
% The {geometry} package allows for margin customization.
% \usepackage{geometry}
% The {fancyhdr} package allows for header/footer customization.
% \usepackage{fancyhdr}
%\pagestyle{fancy}
% The {titlesec} packages allows for altering section/pages
% commands/styles.
%\usepackage{titlesec}
% -- Here, I am redefining a section break to mean that the next section
%  begins
%    on a new, clean page.
%\newcommand{\sectionbreak}{\clearpage}
% - - - - - - - - - - - - - - - - - - - - - - - - - -



% TABLES and FIGURES
% - - - - - - - - - - - - - - - - - - - - - - - - - -
% The {array} package is needed for extra control (might not be
% necessary)
%   \usepackage{array}
% The {multirow} package allows you to have more general tables
%  -- also allows to choose which cells get borders
%  e.g.:    
%  Year | Month |   Days w/ Data   |
%  --------------------------------|
%  2000 |  01   | 03, 04, 10       |
%       ---------------------------|
%       |  06   | 15, 16           |
%        --------------------------|
%       |  11   | 02, 23, 24, 27   |
%  --------------------------------
\usepackage{multirow}
% --btw, for floating tables, tabular environment must be placed 
%   inside a table environment; if you want your table to be 
%   exactly where you placed it, then only use tabular environment
% --the table environment also allows one to add a \caption{}, 
%   specify position (h: here, t: top, b: bottom, p: own page),
%   and use labels (e.g., \label{tab:myTable}) so it can be 
%   referenced later (\ref{tab:myTable})


% The {caption} and {subcaption} packages allow you to create
% multi-panel figures
\usepackage{caption}
\usepackage{subcaption}

% - - - - - - - - - - - - - - - - - - - - - - - - - -





% REFERENCES / BIBLIOGRAPHY
% - - - - - - - - - - - - - - - - - - - - - - - - - -
% The {biblatex} package is used for references
% Default style is numeric; can be changed to alphabetic or authoryear.
% Sorting: none (bib in order refs are cited); ynt (year, name title);
% etc
%\usepackage[style=authoryear,sorting=nyt]{biblatex}
%\addbibresource{references.bib}
% That said,
% for better author/year citing, the package {natbib} is often used
\usepackage[square]{natbib}
% More info: http://en.wikibooks.org/wiki/LaTeX/Bibliography_Management

% For INLINE BIBLIOGRAPHY entries:
% To have inline refs, must write: \nobibliography{<bib file>}
% If you still want bibliography at end, must still use
% \bibliography{..}, and also  \nobibliography*
% For inline reference:  \bibentry{<cite tag>}
\usepackage{bibentry}
% --- if you want to include a CV (e.g., in my dissertation proposal),
%  this is an ok way to do it, however, all CV items will also
%  reappear in the bibliography at the end...which you may or may not
%  care for (this package works by using the .bbl file that generates
%  the bibliograph).
% --- you might create a separate LaTeX project to do a CV, use
%  {bibentry} to get the proper bibliography formats, then copy and
%  paste
%
% - - - - - - - - - - - - - - - - - - - - - - - - - -


% Quick Reference:
% List environments: itemize, enumerate
% Math: equation, align, aligned, gather

% % % % % % % % % % % % % % % % % % % % % % % % % % % % % % % % % % % % % % % % % % % % % % % % 
% % % % % % % % % % % % % % % % % % % % % % % % % % % % % % % % % % % % % % % % % % % % % % % % 
% % % % % % % % % % % % % % % % % % % % % % % % % % % % % % % % % % % % % % % % % % % % % % % % 
% % % % % % % % % % % % % % % % % % % % % % % % % % % % % % % % % % % % % % % % % % % % % % % % 
% % % % % % % % % % % % % % % % % % % % % % % % % % % % % % % % % % % % % % % % % % % % % % % % 
% % % % % % % % % % % % % % % % % % % % % % % % % % % % % % % % % % % % % % % % % % % % % % % % 
%                              #=#=# END PREAMBLE  #=#=#
% % % % % % % % % % % % % % % % % % % % % % % % % % % % % % % % % % % % % % % % % % % % % % % % 
