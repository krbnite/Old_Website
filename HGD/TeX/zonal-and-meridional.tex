
When interpreting vectorial data on sphere, one
often uses a localized spherical coordinate system, as opposed to a
static Cartesian system. This is because the local coordinate system is
more intuitive when discussing and interpreting the data; further more,
the coordinates have the same meaning everywhere. This is in contrast to
using a static Cartesian system, where the components at each point on
there sphere, although along the same axes everywhere, take on different
meaning. 

For example, imagine the Z axis is defined to point in a direction
parallel to the rotation axis, the X axis points orthogonally out of the
middle of North America, and the Y axis complete the orthogonal
coordinate system. At the North pole, the vertical direction (``up'') is the Z
direction, but in the middle of North America, ``up'' is in the X
direction. 

In a local coordinate system, instead of choosing static basis vectors,
we choose to instead allow the basis vectors to change while holding
directional concepts constant.

So in spherical coordinates, for one, we have the radial (or vertical)
component. This means no matter where on the Earth's surface or in its
atmosphere, when I talk about the vertical drift, you know exactly what
I'm talking about. Imagine using a Cartesian system and talking about
the Y drift at multiple locations on the sphere? At each location, it
refers to a different drift: sometimes vertically up or vertically down,
sometimes purely horizontal, and other times some mix of the two.

MERIDIONAL (N-S)
We also have the polar angle component. Note that this is defined in
reference to background, static Cartesian coordinate system. That said,
it is not confusing: if the Z axis of the Cartesian system is the
rotation axis, then the polar angle component is simply the Southward
component at a point. This polar component in the local system is also
referred to as the meridional component in many areas of geophysics.

ZONAL (E-W)
In a local spherical basis, we also have the azimuthal component. This
component represents the component of the flow parallel to the XY plane
in the background, static cartesian system. Or, another way of saying
this is that the azimuthal component is the non-vertical component that
is also perpendicular to the meridional component. This azimuthal
component is often called the zonal component in geophysical
applications.


